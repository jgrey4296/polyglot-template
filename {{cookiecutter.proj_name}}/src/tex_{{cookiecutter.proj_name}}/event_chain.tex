%% compiler:lualatex
%% available compilers : lualatex pdflatex xelatex
% !TeX TS-program = lualatex
% !TeX encoding = UTF-8
\documentclass[class=article, crop=false]{standalone}

%%-- imports
\usepackage{iftex}
\usepackage{fontspec}
\usepackage[colorlinks=true, linkcolor=blue, urlcolor=blue, pdfstartview=FitH]{hyperref}
\usepackage[numbers]{natbib}
\usepackage[russian,english]{babel}
\usepackage[version=3]{mhchem}
\usepackage{doi}
\usepackage{listings}
\usepackage{url}
\usepackage{lettrine}
\usepackage{titling}
\usepackage{pdflscape}
\usepackage{geometry}
%%-- end imports

\usepackage{responsive-title}
\usepackage{eventchain}

\def\eventChainTitle{Event Chains Example}
\author{John Grey}
\date{21-09-2025}
\setlength{\parskip}{2em}

\tikzset{
  /eventchain/node/.style={draw,circle,font=\tiny},
  % /eventchain/event/node/.style={.default=/eventchain/node, fill=blue!20, shape=diamond},
  % /eventchain/event/box/.style={-latex, fill=blue!20, shape=rectangle},
  % /eventchain/state/node/.style={.default=/eventchain/node, fill=red!21},
  % /eventchain/state/box/.style={-latex, fill=red!20, shape=rectangle, rounded corners},
  % /eventchain/skip/node/.style={.default=/eventchain/node, draw=none},
}

\begin{document}
\resptitle{\eventChainTitle}{\eventChainTitle}
\renewcommand{\abstractname}{}
\begin{abstract}
Examples of the use of `eventchain'. Both vertical and horizontal. 
\end{abstract}

\hrulefill
\vfill
\pagebreak

%% ---------- Content ----------
\begin{figure}
  \begin{center}
  \begin{eventchain}[start=5, dist=0.5cm]
    \state
    \begin{event}[name=First Event, dist=1cm, cols=2, fmt={c | l}]
      bloo ($ \alpha \omega $) \\
    \end{event}
    \begin{fluents}[name=Fluents, dist=2cm, cols=2, fmt={c | l}]
      \addfluent{blah} \\
      \subfluent{bloo($ \alpha \omega $)} \\
      \keepfluent{aweg} \\
    \end{fluents}
    \begin{event}[name=Second Event, cols=2, fmt={c | l}]
      \subfluent{aweg} \\
      awegjoi \\
    \end{event}
    \begin{event}[name=Third Event]
      baweg \\
      awegjoi \\
    \end{event}
  \end{eventchain}
  \end{center}
  \caption{A Top-Bottom example}
\end{figure}

\pagebreak

\begin{landscape}
  \begin{figure}[th]
    \caption{A Left-Right Example}
    \begin{center}
    \begin{eventchain}[dir=right, dist=0.2cm]
      \state[5]
      \state
      \state
      \state
      \state
      % \event
      \begin{event}[name=Event, dist=0.7cm]
        blah \\
        aweg \\
      \end{event}
      \state
      \begin{fluents}[name=State, dist=0.7cm, cols=2, fmt={c | l}]
        \addfluent{blah} \\
        \subfluent{bloo($ \alpha \omega $)} \\
      \end{fluents}
      \state
      \state
      \state
      \jumpto[100]
    \end{eventchain}
    \end{center}
  \end{figure}

  \hrulefill

  \begin{figure}[bh]
    \begin{center}
    \begin{eventchain}[dir=right, dist=0.2cm]
      \state[5]
      \state
      \state
      \jumpto[10]
      \state
      \state
      % \event
      \begin{event}[name=Event, dist=0.7cm]
        blah \\
        bloo($ \alpha \omega $) \\
      \end{event}
      \state
      \begin{fluents}[name=State, dist=0.7cm, cols=2, fmt={c | l}]
        \addfluent{blah} \\
        \subfluent{bloo($ \alpha \omega $)} \\
      \end{fluents}
      \state
      \state
      \state
      \jumpto[100]
    \end{eventchain}
    \end{center}
    \caption{A Left-Right Example}
  \end{figure}
\end{landscape}
\restoregeometry

%% ---------- Footnotes ----------

% \footnotemark[N]
% \footnotetext[N]{Cras placerat accumsan nulla.}

%% --------------------------------------------------
\end{document}
